% Options for packages loaded elsewhere
\PassOptionsToPackage{unicode}{hyperref}
\PassOptionsToPackage{hyphens}{url}
\PassOptionsToPackage{dvipsnames,svgnames,x11names}{xcolor}
%
\documentclass[
  a4paper,
]{article}

\usepackage{amsmath,amssymb}
\usepackage{iftex}
\ifPDFTeX
  \usepackage[T1]{fontenc}
  \usepackage[utf8]{inputenc}
  \usepackage{textcomp} % provide euro and other symbols
\else % if luatex or xetex
  \usepackage{unicode-math}
  \defaultfontfeatures{Scale=MatchLowercase}
  \defaultfontfeatures[\rmfamily]{Ligatures=TeX,Scale=1}
\fi
\usepackage{lmodern}
\ifPDFTeX\else  
    % xetex/luatex font selection
\fi
% Use upquote if available, for straight quotes in verbatim environments
\IfFileExists{upquote.sty}{\usepackage{upquote}}{}
\IfFileExists{microtype.sty}{% use microtype if available
  \usepackage[]{microtype}
  \UseMicrotypeSet[protrusion]{basicmath} % disable protrusion for tt fonts
}{}
\makeatletter
\@ifundefined{KOMAClassName}{% if non-KOMA class
  \IfFileExists{parskip.sty}{%
    \usepackage{parskip}
  }{% else
    \setlength{\parindent}{0pt}
    \setlength{\parskip}{6pt plus 2pt minus 1pt}}
}{% if KOMA class
  \KOMAoptions{parskip=half}}
\makeatother
\usepackage{xcolor}
\setlength{\emergencystretch}{3em} % prevent overfull lines
\setcounter{secnumdepth}{5}
% Make \paragraph and \subparagraph free-standing
\makeatletter
\ifx\paragraph\undefined\else
  \let\oldparagraph\paragraph
  \renewcommand{\paragraph}{
    \@ifstar
      \xxxParagraphStar
      \xxxParagraphNoStar
  }
  \newcommand{\xxxParagraphStar}[1]{\oldparagraph*{#1}\mbox{}}
  \newcommand{\xxxParagraphNoStar}[1]{\oldparagraph{#1}\mbox{}}
\fi
\ifx\subparagraph\undefined\else
  \let\oldsubparagraph\subparagraph
  \renewcommand{\subparagraph}{
    \@ifstar
      \xxxSubParagraphStar
      \xxxSubParagraphNoStar
  }
  \newcommand{\xxxSubParagraphStar}[1]{\oldsubparagraph*{#1}\mbox{}}
  \newcommand{\xxxSubParagraphNoStar}[1]{\oldsubparagraph{#1}\mbox{}}
\fi
\makeatother


\providecommand{\tightlist}{%
  \setlength{\itemsep}{0pt}\setlength{\parskip}{0pt}}\usepackage{longtable,booktabs,array}
\usepackage{calc} % for calculating minipage widths
% Correct order of tables after \paragraph or \subparagraph
\usepackage{etoolbox}
\makeatletter
\patchcmd\longtable{\par}{\if@noskipsec\mbox{}\fi\par}{}{}
\makeatother
% Allow footnotes in longtable head/foot
\IfFileExists{footnotehyper.sty}{\usepackage{footnotehyper}}{\usepackage{footnote}}
\makesavenoteenv{longtable}
\usepackage{graphicx}
\makeatletter
\newsavebox\pandoc@box
\newcommand*\pandocbounded[1]{% scales image to fit in text height/width
  \sbox\pandoc@box{#1}%
  \Gscale@div\@tempa{\textheight}{\dimexpr\ht\pandoc@box+\dp\pandoc@box\relax}%
  \Gscale@div\@tempb{\linewidth}{\wd\pandoc@box}%
  \ifdim\@tempb\p@<\@tempa\p@\let\@tempa\@tempb\fi% select the smaller of both
  \ifdim\@tempa\p@<\p@\scalebox{\@tempa}{\usebox\pandoc@box}%
  \else\usebox{\pandoc@box}%
  \fi%
}
% Set default figure placement to htbp
\def\fps@figure{htbp}
\makeatother
% definitions for citeproc citations
\NewDocumentCommand\citeproctext{}{}
\NewDocumentCommand\citeproc{mm}{%
  \begingroup\def\citeproctext{#2}\cite{#1}\endgroup}
\makeatletter
 % allow citations to break across lines
 \let\@cite@ofmt\@firstofone
 % avoid brackets around text for \cite:
 \def\@biblabel#1{}
 \def\@cite#1#2{{#1\if@tempswa , #2\fi}}
\makeatother
\newlength{\cslhangindent}
\setlength{\cslhangindent}{1.5em}
\newlength{\csllabelwidth}
\setlength{\csllabelwidth}{3em}
\newenvironment{CSLReferences}[2] % #1 hanging-indent, #2 entry-spacing
 {\begin{list}{}{%
  \setlength{\itemindent}{0pt}
  \setlength{\leftmargin}{0pt}
  \setlength{\parsep}{0pt}
  % turn on hanging indent if param 1 is 1
  \ifodd #1
   \setlength{\leftmargin}{\cslhangindent}
   \setlength{\itemindent}{-1\cslhangindent}
  \fi
  % set entry spacing
  \setlength{\itemsep}{#2\baselineskip}}}
 {\end{list}}
\usepackage{calc}
\newcommand{\CSLBlock}[1]{\hfill\break\parbox[t]{\linewidth}{\strut\ignorespaces#1\strut}}
\newcommand{\CSLLeftMargin}[1]{\parbox[t]{\csllabelwidth}{\strut#1\strut}}
\newcommand{\CSLRightInline}[1]{\parbox[t]{\linewidth - \csllabelwidth}{\strut#1\strut}}
\newcommand{\CSLIndent}[1]{\hspace{\cslhangindent}#1}

\makeatletter
\@ifpackageloaded{caption}{}{\usepackage{caption}}
\AtBeginDocument{%
\ifdefined\contentsname
  \renewcommand*\contentsname{Table of contents}
\else
  \newcommand\contentsname{Table of contents}
\fi
\ifdefined\listfigurename
  \renewcommand*\listfigurename{List of Figures}
\else
  \newcommand\listfigurename{List of Figures}
\fi
\ifdefined\listtablename
  \renewcommand*\listtablename{List of Tables}
\else
  \newcommand\listtablename{List of Tables}
\fi
\ifdefined\figurename
  \renewcommand*\figurename{Figure}
\else
  \newcommand\figurename{Figure}
\fi
\ifdefined\tablename
  \renewcommand*\tablename{Table}
\else
  \newcommand\tablename{Table}
\fi
}
\@ifpackageloaded{float}{}{\usepackage{float}}
\floatstyle{ruled}
\@ifundefined{c@chapter}{\newfloat{codelisting}{h}{lop}}{\newfloat{codelisting}{h}{lop}[chapter]}
\floatname{codelisting}{Listing}
\newcommand*\listoflistings{\listof{codelisting}{List of Listings}}
\makeatother
\makeatletter
\makeatother
\makeatletter
\@ifpackageloaded{caption}{}{\usepackage{caption}}
\@ifpackageloaded{subcaption}{}{\usepackage{subcaption}}
\makeatother

\ifLuaTeX
\usepackage[bidi=basic]{babel}
\else
\usepackage[bidi=default]{babel}
\fi
\babelprovide[main,import]{british}
% get rid of language-specific shorthands (see #6817):
\let\LanguageShortHands\languageshorthands
\def\languageshorthands#1{}
\ifLuaTeX
  \usepackage[english]{selnolig} % disable illegal ligatures
\fi
\usepackage{bookmark}

\IfFileExists{xurl.sty}{\usepackage{xurl}}{} % add URL line breaks if available
\urlstyle{same} % disable monospaced font for URLs
\hypersetup{
  pdftitle={Is reproducibility good enough?},
  pdfauthor={Ivan Stordal},
  pdflang={en-GB},
  colorlinks=true,
  linkcolor={blue},
  filecolor={Maroon},
  citecolor={Blue},
  urlcolor={Blue},
  pdfcreator={LaTeX via pandoc}}


\title{Is reproducibility good enough?}
\author{Ivan Stordal}
\date{Tuesday 23 Sep, 2025}

\begin{document}
\maketitle
\begin{abstract}
The credibility of science depends on whether research findings are
robust, reliable, and reproducible. Over the past two decades, several
fields have faced what is now called the ``replication crisis'' where
publication bias, small samples, and flexible analysis practices have
weakened confidence in results. This paper builds on the lecture Robust
and Reliable Science and reviews literature on reproducibility and
replicability, highlighting how large replication studies and
methodological critiques have revealed shortcomings in current
practices. The paper argues that computational reproducibility should be
treated as a basic requirement for credibility, while full replication
is more demanding and often limited by resources. Tools such as Quarto
and GitHub, combined with journal policies, preregistration, and better
training in open science practices, provide promising ways forward.
Although progress has been made, reproducibility is still not
sufficient. Stronger standards and incentives are needed to move science
toward greater robustness and reliability.
\end{abstract}


\section{Introduction}\label{introduction}

The trustworthiness of research relies on whether results can be
reproduced and verified. In the last two decades, many disciplines have
faced growing concern over the ``replication crisis'', raising doubts
about the reliability of published findings (Ioannidis, 2005).
Reproducibility refers to obtaining the same results using the original
data and methods, while replicability means validating those results
with new data (Goodman et al., 2016). Both are essential for cumulative
science, yet evidence shows that current practices often fall short.

This paper asks whether reproducibility today is ``good enough.''
Drawing on the lecture Robust and Reliable Science and key sources, it
reviews problems such as publication bias, Type I errors, and
replication failures in psychology and economics. It then discusses
potential solutions including preregistration, data and code sharing,
and technical tools such as Quarto. The goal is to evaluate the current
state of reproducibility and outline what is still needed to make
science more reliable.

\section{Literature review}\label{literature-review}

The distinction between reproducibility and replicability has become
central to debates about research quality. Reproducibility means running
the same analysis on the same data to obtain the same result, while
replicability requires re-testing findings on new material (Goodman et
al., 2016; Peng, 2011). Scholars often describe reproducibility as the
baseline for credible work, with replicability as the gold standard
(Munafò et al., 2017).

One of the main barriers to reliability is publication bias, sometimes
called the ``file drawer problem'' (Rosenthal, 1979). Positive and
surprising results are more likely to be published, which creates a
distorted scientific record. Simmons et al. (2011) point out that this
issue is worsened by the risk of Type I errors, especially when
researchers have in how they collect or analyze data. Studies show that
these practices inflate reported effect sizes, reducing the
trustworthiness of the literature (Button et al., 2013).

Large replication projects illustrate the scale of the problem. The Open
Science Collaboration (2015) tried to replicate 100 studies in
psychology and succeeded in reproducing only about one third. In
economics, Dewald et al. (1986) showed that many empirical results could
not be reproduced because data or code were missing. They used data from
the 1982 \emph{JMCB Data Storage and Evaluation Project}.

Proposed solutions include stronger journal policies requiring authors
to provide code and data, sometimes linked to permanent repositories
(e.g., Barnes, 2010; Bechhofer et al., 2013). Complementary reforms such
as preregistration and registered reports aim to reduce questionable
research practices (Nosek et al., 2015). Technical frameworks also play
a role: literate programming (Gentleman, 2005; Xie, 2015), practical
guidelines for computational reproducibility (Sandve et al., 2013), and
newer publishing systems like Quarto (Allaire et al., 2020) make it
easier to integrate code, data, and text into workflows.

\section{Discussion of the reseach
question}\label{discussion-of-the-reseach-question}

Should replicability be the norm?

Most researchers agree that computational reproducibility should be the
minimum requirement for published work (Peng, 2011). Replication,
however, is more expensive and not always feasible, especially in large
or sensitive studies. A pragmatic approach is to enforce reproducibility
across the board, while prioritizing replication in high-stakes areas
such as clinical trials or economic policy (Munafò et al., 2017).

Can Quarto documents help?

Quarto provides a promising way to make research workflows more
transparent. By combining code, narrative, and results in one document,
it reduces the risk of errors from copy-pasting (Wickham \& Grolemund,
2016; Xie, 2020). When linked with GitHub, it also makes collaboration
and peer review easier. Experience from teaching shows that adopting
such tools can lower the barrier to working openly and reproducibly,
especially for students and early-career researchers (Broman \& Woo,
2018).

What problems remain?

Despite technical progress, cultural and institutional barriers remain.
Incentives still reward novelty over careful replication, making it less
attractive to reproduce others' work (Begley \& Ellis, 2012). Small
samples and flexible analytic choices continue to generate unstable
results (Button et al., 2013). Even when journals require data and code,
enforcement is inconsistent (Miguel et al., 2014).

Several concrete steps could help:

\begin{description}
\item[Policy and enforcement]
stricter journal checks for code and data availability.
\item[Preregistration and registered reports]
limiting bias by committing to methods before results are known (Nosek
et al., 2015).
\item[Training and infrastructure]
promoting reproducible pipelines and collaborative platforms (Sandve et
al., 2013).
\item[Environment capture]
using tools such as renv in R to freeze dependencies and ensure analyses
remain reproducible over time (Starr et al., 2015).
\end{description}

\section{Conclusion}\label{conclusion}

The reproducibility debate shows both progress and persistent
challenges. Psychology, economics, and biomedicine all provide evidence
that many results cannot be reproduced (Begley \& Ellis, 2012;
Collaboration, 2015; Dewald et al., 1986). At the same time, reforms
such as preregistration, journal data policies, and computational tools
are helping to improve research practices (Munafò et al., 2017; Sandve
et al., 2013).

The overall conclusion is that reproducibility is not yet good enough.
Technical obstacles have been reduced, but incentives and enforcement
still lag behind. A realistic path forward is to require computational
reproducibility for all published work, while encouraging replication
where the stakes are high. With wider use of Quarto, GitHub, and data
archives, science can continue moving toward being more robust and
reliable.

\section{Software used}\label{software-used}

\subsection{R version}\label{r-version}

R version 4.5.1 (2025-06-13)

\subsection{Attached packages:}\label{attached-packages}

attached base packages: {[}1{]} stats, graphics, grDevices, utils,
datasets, methods, base,

other attached packages: {[}1{]} lubridate\_1.9.4, forcats\_1.0.0,
stringr\_1.5.1, dplyr\_1.1.4, purrr\_1.1.0,\\
{[}6{]} readr\_2.1.5, tidyr\_1.3.1, tibble\_3.3.0, ggplot2\_3.5.2, ---

\section*{References}\label{references}
\addcontentsline{toc}{section}{References}

\phantomsection\label{refs}
\begin{CSLReferences}{1}{0}
\bibitem[\citeproctext]{ref-allaire2020b}
Allaire, J., Xie, Y., McPherson, J., Luraschi, J., Ushey, K., Atkins,
A., Wickham, H., Cheng, J., Chang, W., \& Iannone, R. (2020).
\emph{Rmarkdown: {Dynamic} documents for r}.

\bibitem[\citeproctext]{ref-barnes2010b}
Barnes, N. (2010). Publish your computer code: It is good enough.
\emph{Nature}, \emph{467}(7317), 753--753.

\bibitem[\citeproctext]{ref-bechhofer2013b}
Bechhofer, S., Buchan, I., De Roure, D., Missier, P., Ainsworth, J.,
Bhagat, J., Couch, P., Cruickshank, D., Delderfield, M., Dunlop, I.,
Gamble, M., Michaelides, D., Owen, S., Newman, D., Sufi, S., \& Goble,
C. (2013). Why linked data is not enough for scientists. \emph{Future
Generation Computer Systems}, \emph{29}(2), 599--611.

\bibitem[\citeproctext]{ref-begley2012}
Begley, C. G., \& Ellis, L. M. (2012). Raise standards for preclinical
cancer research. \emph{Nature}, \emph{483}(7391), 531--533.

\bibitem[\citeproctext]{ref-broman2018}
Broman, K. W., \& Woo, K. H. (2018). \emph{Data organization in
spreadsheets} (No. e3183v2). PeerJ Inc.

\bibitem[\citeproctext]{ref-button2013}
Button, K. S., Ioannidis, J. P. A., Mokrysz, C., Nosek, B. A., Flint,
J., Robinson, E. S. J., \& Munafò, M. R. (2013). Power failure: Why
small sample size undermines the reliability of neuroscience.
\emph{Nature Reviews Neuroscience}, \emph{14}(5), 365--376.

\bibitem[\citeproctext]{ref-opensciencecollaboration2015}
Collaboration, O. S. (2015). Estimating the reproducibility of
psychological science. \emph{Science}, \emph{349}(6251).

\bibitem[\citeproctext]{ref-dewald1986b}
Dewald, W. G., Thursby, J. G., \& Anderson, R. G. (1986). Replication in
{Empirical Economics}: {The Journal} of {Money}, {Credit} and {Banking
Project}. \emph{The American Economic Review}, \emph{76}(4), 587--603.

\bibitem[\citeproctext]{ref-gentleman2005b}
Gentleman, R. (2005). Reproducible {Research}: {A Bioinformatics Case
Study}. \emph{Statistical Applications in Genetics and Molecular
Biology}, \emph{4}(1).

\bibitem[\citeproctext]{ref-goodman2016b}
Goodman, S. N., Fanelli, D., \& Ioannidis, J. P. A. (2016). What does
research reproducibility mean? \emph{Science Translational Medicine},
\emph{8}(341), 341ps12--341ps12.

\bibitem[\citeproctext]{ref-ioannidis2005b}
Ioannidis, J. P. A. (2005). Why {Most Published Research Findings Are
False}. \emph{PLOS Medicine}, \emph{2}(8), e124.

\bibitem[\citeproctext]{ref-miguel2014}
Miguel, E., Camerer, C., Casey, K., Cohen, J., Esterling, K. M., Gerber,
A., Glennerster, R., Green, D. P., Humphreys, M., Imbens, G., Laitin,
D., Madon, T., Nelson, L. D., Nosek, B. A., Petersen, M., Sedlmayr, R.,
Simmons, J. P., \& Simonsohn, U. (2014). Promoting transparency in
social science research. \emph{Science}, \emph{343}(6166), 30--31.

\bibitem[\citeproctext]{ref-munafo2017}
Munafò, M. R., Nosek, B. A., Bishop, D. V., Button, K. S., Chambers, C.
D., Sert, N. P. du, Simonsohn, U., Wagenmakers, E.-J., Ware, J. J., \&
Ioannidis, J. P. (2017). A manifesto for reproducible science.
\emph{Nature Human Behaviour}, \emph{1}, 0021.

\bibitem[\citeproctext]{ref-nosek2015}
Nosek, B. A., Alter, G., Banks, G. C., Borsboom, D., Bowman, S. D.,
Breckler, S. J., Buck, S., Chambers, C. D., Chin, G., Christensen, G.,
Contestabile, M., Dafoe, A., Eich, E., Freese, J., Glennerster, R.,
Goroff, D., Green, D. P., Hesse, B., Humphreys, M., \ldots{} Yarkoni, T.
(2015). Promoting an open research culture. \emph{Science},
\emph{348}(6242), 1422--1425.

\bibitem[\citeproctext]{ref-peng2011}
Peng, R. D. (2011).
\href{https://www.ncbi.nlm.nih.gov/pubmed/22144613}{Reproducible
{Research} in {Computational Science}}. \emph{Science},
\emph{334}(6060), 1226--1227.

\bibitem[\citeproctext]{ref-rosenthal1979}
Rosenthal, R. (1979). \emph{The file drawer problem and tolerance for
null results.}

\bibitem[\citeproctext]{ref-sandve2013}
Sandve, G. K., Nekrutenko, A., Taylor, J., \& Hovig, E. (2013). Ten
simple rules for reproducible computational research. \emph{PLoS
Computational Biology}, \emph{9}(10), e1003285.

\bibitem[\citeproctext]{ref-simmons2011}
Simmons, J. P., Nelson, L. D., \& Simonsohn, U. (2011). False-positive
psychology: {Undisclosed} flexibility in data collection and analysis
allows presenting anything as significant. \emph{Psychological Science},
\emph{22}(11), 1359--1366.

\bibitem[\citeproctext]{ref-starr2015}
Starr, J., Castro, E., Crosas, M., Dumontier, M., Downs, R. R., Duerr,
R., Haak, L. L., Haendel, M., Herman, I., Hodson, S., Hourclé, J.,
Kratz, J. E., Lin, J., Nielsen, L. H., Nurnberger, A., Proell, S.,
Rauber, A., Sacchi, S., Smith, A., \ldots{} Clark, T. (2015). Achieving
human and machine accessibility of cited data in scholarly publications.
\emph{PeerJ Computer Science}, \emph{1}, e1.

\bibitem[\citeproctext]{ref-wickham2016}
Wickham, H., \& Grolemund, G. (2016). \emph{R for data science: Import,
tidy, transform, visualize, and model data} (pp. XXV, 492). O'Reilly.

\bibitem[\citeproctext]{ref-xie2015}
Xie, Y. (2015). \emph{Dynamic documents with {R} and knitr} (Second).
{Chapman and Hall/CRC}.

\bibitem[\citeproctext]{ref-xie2020}
Xie, Y. (2020). \emph{Knitr: {A} general-purpose package for dynamic
report generation in r} {[}Manual{]}.

\end{CSLReferences}




\end{document}
